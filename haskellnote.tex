\documentclass[platex,a5paper,twoside,fleqn,draft]{jsbook}
\usepackage{graphicx} % Required for inserting images
\usepackage{amsfonts,amssymb,amsmath}
\usepackage{latexsym}
\usepackage{fancybox}
\usepackage{MnSymbol}
\usepackage{stmaryrd}
\usepackage{manfnt}
\usepackage{bm}
% \usepackage{tikz}
% \usetikzlibrary{graphs}
\usepackage{haskellstyle}

\usepackage{geometry}
\usepackage{ascmac}

\title{Haskell Notes v.0}
\author{Ichi Kanaya}
\date{2025}

\newcommand{\keyword}[1]{\textgt{#1}}
\newcommand{\tobewritten}[1]{\begin{screen}\textbf{TK.} {#1}\end{screen}}

\begin{document}

\maketitle

\section{変数}

変数$x$に値を代入するには次のようにする.\footnote{\haskell では \verb|x = 1| と書く.}
\begin{equation}
x=1
\end{equation}
変数という呼び名に反して,変数の値は変えられない.そこで変数に値を代入するとは呼ばずに,変数名に値を\keyword{束縛}するという.

変数の値がいつでも変化しないことを\keyword{参照透過性}と呼ぶ.プログラマーが変数の値を変化させたい理由はユーザー入力,ループ,例外,内部状態,大域ジャンプ,継続を扱いたいからであろう.しかし,後に見るようにループ,例外,内部状態,大域ジャンプ,継続に変数の破壊的代入は必要ない.ユーザー入力に関しても章を改めて取り上げる.

本書では変数名を原則1文字として,イタリック体で表し,$w,x,y,z$のような$n$以降のアルファベットを使う.

\section{関数}

関数$f$は次のように定義できる.\footnote{\haskell では \verb|f x = x+1| と書く.}
\begin{equation}
fx=x+1
\end{equation}
ここに$x$は関数$f$の引数である.引数は括弧でくるまない.

本書では関数名を原則1文字として,イタリック体で表し,$f,g,h$のようにアルファベットの$f$以降の文字を使う.ただし有名な関数についてはローマン体で表し,文字数も2文字以上とする.たとえば$\sin$などの三角関数や指数関数がそれにあたる.

変数$x$に関数$f$を\keyword{適用}する場合は次のように書く.\footnote{\haskell では \verb|z = f x| と書く.}
\begin{equation}
z=fx
\end{equation}

関数$f$が引数をふたつ取る場合は,次のように書く.\footnote{\haskell では \verb|z = f x y| と書く.}
\begin{equation}
z=fxy
\end{equation}
なお$fxy$は$(fx)y$と解釈される.前半の$(fx)$は1引数の関数とみなせる.

\section{ラムダ}

関数とは,変数名に束縛された\keyword{ラムダ式}である.ラムダ式は次のように書く.\footnote{\haskell では \verb|f = \x -> x+1| と書く.}
\begin{equation}
f=\mathLambda{x}{x+1}
\end{equation}

本書では無名変数$\mathAnonymousParameter$を用いた以下の書き方も用いる.\footnote{無名変数は\haskell には無いが,代わりに「セクション」という書き方ができる.式$(\mathAnonymousParameter+1)$は\haskell では\verb|(+1)| と書く.}
\begin{align}
f&=\left(\mathAnonymousParameter+1\right)\\
&=\mathLambda{x}{x+1}
\end{align}

\section{ローカル変数}

関数内で\keyword{ローカル変数}を使いたい場合は以下のように行う.\footnote{\haskell では \verb|z = let {y = 1} in x+y| と書く.\verb|let| 節内の式がひとつの場合,中括弧は省略可能である.式が複数になる場合は \verb|;| で区切る.}
\begin{equation}
z=\mathLetIn{y=1}{x+y}
\end{equation}

ローカル変数の定義は次のように後置できる.\footnote{\haskell では \verb|z = x+y where {y = 1}| と書く.\verb|where| 節内の式が一つの場合,中括弧は省略可能である.式が複数になる場合は \verb|;| で区切る.}
\begin{equation}
z=x+y\mathWhere{y=1}
\end{equation}

\section{クロージャ}

ラムダ式を返す関数は,ラムダ式内部に値を閉じ込めることができる.
\begin{equation}
fn=\mathLambda{x}{n+x}
\end{equation}
関数$f$に引数$n$を与えると,新たな1引数関数が得られる.例を挙げる.
\begin{align}
n&=3\\
g&=fn
\end{align}
この例では,関数$g$の中に値$n=3$が閉じ込められているため$g1$は$4$と評価される.値を閉じ込めたラムダ式を\keyword{クロージャ}と呼ぶ.

\section{型}

すべての変数,関数には\keyword{型}がある.代表的な型にはブール型,整数型,浮動小数点型,文字型がある.以降,ブール型を$\mathTypeBool$で,整数型を$\mathTypeInt$で表す.\footnote{\haskell ではブール型を \verb|Bool|,整数型を \verb|Int|,多倍長整数型を \verb|Integer| と書く.}

浮動小数点型は実数全体を表現できないが,本書では実数全体を意味する$\mathTypeReal$で表すことにする.\footnote{\haskell では単精度浮動小数点型を \verb|Float|,倍精度浮動小数点型を \verb|Double| と書く.}

本書では対応する,あるいは近い数学概念がある場合,型名をブラックボード体1文字で書く.文字型のように対応する数学概念がない場合はボールドローマン体を用いる.文字型は$\mathTypeChar$とする.\footnote{\haskell ではUnicode文字型を \verb|Char| と書く.}

変数$x$の型が$\mathTypeInt$のとき,以下のように\keyword{型注釈}を書く.\footnote{\haskell では \verb|x :: Int| と書く.}
\begin{equation}
x\mathTypeIs\mathTypeInt
\end{equation}
同じことを数学者は$x\in\mathTypeInt$と書くことを好むが,記号$\in$は別の用途で使うため$\mathTypeIs$を用いる.

1引数関数の型は次のように注釈できる.\footnote{\haskell では \verb|f :: Int -> Int| と書く.}
\begin{equation}
f\mathTypeIs\mathTypeFunction{\mathTypeInt}{\mathTypeInt}
\end{equation}
ここで関数$f$は整数型の引数をひとつとり,整数型の値を返す.

2引数関数の方は次のように注釈できる.\footnote{\haskell では \verb|f :: Int -> Int -> Int| と書く.}
\begin{equation}
f\mathTypeIs\mathTypeFunctionII{\mathTypeInt}{\mathTypeInt}{\mathTypeInt}
\end{equation}
ここで関数$f$は整数型の引数をふたつとり,整数型の値を返す.型$\mathTypeFunctionII{\mathTypeInt}{\mathTypeInt}{\mathTypeInt}$は$\mathTypeFunction{\mathTypeInt}{\left(\mathTypeFunction{\mathTypeInt}{\mathTypeInt}\right)}$と解釈される.

$(\mathTypeFunction{\mathTypeInt}{\mathTypeInt})$型の関数を受け取り,$(\mathTypeFunction{\mathTypeInt}{\mathTypeInt})$型の関数を返す関数は次の型を持つ.\footnote{\haskell では以下のように書く.
\begin{footcode}
      f :: (Int -> Int) -> (Int -> Int)
\end{footcode}}
\begin{equation}
f\mathTypeIs\mathTypeFunction{\left(\mathTypeFunction{\mathTypeInt}{\mathTypeInt}\right)}
  {\left(\mathTypeFunction{\mathTypeInt}{\mathTypeInt}\right)}
\end{equation}
なお後半の括弧は省略可能なので
\begin{equation}
f\mathTypeIs\mathTypeFunction{\left(\mathTypeFunction{\mathTypeInt}{\mathTypeInt}\right)}
  {\mathTypeFunction{\mathTypeInt}{\mathTypeInt}}
\end{equation}
と書いても良い.

\section{リテラル}

\keyword{リテラル}は次のように書く.\footnote{\haskell では \verb|z = 1| と書く.}
\begin{equation}
z=1
\end{equation}

\section{条件}

\keyword{条件分岐}は次のように書く.\footnote{\haskell では \verb|z = if x>0 then x else -x| と書く.}
\begin{equation}
z=\mathIf{x>0}{x}{-x}
\end{equation}

条件分岐の代わりに以下のような\keyword{パターンマッチ}も使える.\footnote{\haskell では以下のように書くのが一般的である.
\begin{footcode}
      f = case x of 1 -> 1
                    _ -> 0
\end{footcode}}
\begin{equation}
f=\mathCase{x}{1}{1}{\mathAny}{0}
\end{equation}
この場合$x\mathCompareEq1$ならば$f$は$1$を,そうでなければ$f$は$0$を返す.ここに$\mathAny$はすべてのパターンに一致する記号である.パターンマッチは上から順に行われる.

関数定義にもパターンマッチを使える.\footnote{\haskell では次のように書く.
\begin{footcode}
      f 1 = 1
      f _ = 0
\end{footcode}}
\begin{equation}
\mathPatternMatch{f1}{1}{f\mathAny}{0}
\end{equation}

関数定義には次のように\keyword{ガード}と呼ばれる条件を付与することができる.\footnote{\haskell では次のように書く.
\begin{footcode}
      f x | x > 0     = x
          | otherwise = -x
\end{footcode}}
\begin{equation}
\mathPatternMatchConditional{fx}{x>0}{x}{\mathOtherwise}{-x}
\end{equation}

\section{関数の再帰呼び出し}

関数は再帰的に呼び出せる.$n\ge0$を前提とすると,$n$番目のフィボナッチ数を計算する関数$\mathFib$を次のように定義できる.\footnote{\haskell では次のように書く.ただし\haskell には符号なし整数型がないために \verb|n| が正であることを別に担保する必要がある.またこのコードは無駄な再帰呼び出しを行っており実用的ではない.
\begin{footcode}
      fib 0 = 0
      fib 1 = 1
      fib n = fib (n-1) + fib (n-2)
\end{footcode}}
\begin{equation}
\left\{
\begin{aligned}
\mathFib 0&=0\\
\mathFib 1&=1\\
\mathFib n&=\mathFib(n-1)+\mathFib(n-2)
\end{aligned}
\right.
\end{equation}

\section{関数合成}

関数の\keyword{合成}は次のように書く.\footnote{\haskell では \verb|k = g.f| と書く.}
\begin{equation}
k=g\mathCompose f
\end{equation}
関数合成演算子$\mathCompose$は以下のように右結合する.
\begin{align}
f&=f_3\mathCompose f_2\mathCompose f_1\\
&=\left(f_3\mathCompose f_2\right)\mathCompose f_1\\
\end{align}

関数適用のための特別な演算子$\mathApply$があると便利である.演算子$\mathApply$は関数合成演算子よりも優先順位が低い.例を挙げる.\footnote{\haskell では \verb|z = h $ (g.f) x| と書く.}
\begin{align}
z&=h\mathApply(g\mathCompose f)x\\
&=h\left((g\mathCompose f)x\right)
\end{align}

いま任意の関数$f$に対して
\begin{equation}
\mathId f=f
\end{equation}
なる関数$\mathId$があり,かつ任意の関数$f,g,h$に対して
\begin{equation}
(h\mathCompose g)\mathCompose f=h\mathCompose(g\mathCompose f)
\end{equation}
が成り立つとする.このとき関数は\keyword{モノイド}であるという.

\section{タプル}

複数の変数をまとめてひとつの\keyword{タプル}にすることができる.例を挙げる.\footnote{\haskell では \verb|z = (x, y)| と書く.}
\begin{equation}
z=(x,y)
\end{equation}

タプルの型は,要素の型をタプルにしたものである.例えば$\mathTypeInt$が2個からなるタプルの型は次のようになる.\footnote{In \haskell, \verb|z :: (Int, Int)|.}
\begin{equation}
z\mathTypeIs(\mathTypeInt,\mathTypeInt)
\end{equation}

要素を含まないタプルを\keyword{ユニット}と呼ぶ.ユニットは次のように書く.\footnote{\haskell では \verb|z = ()| と書く.}
\begin{equation}
z=()
\end{equation}

ユニットの型は\keyword{ユニット型}で,型注釈を次のように書く.\footnote{\haskell では \verb|z :: ()| と書く.}
\begin{equation}
z\mathTypeIs()
\end{equation}

\section{リストと内包表記}

ある変数が\keyword{リスト}であるとき,その変数がリストであることを忘れないように$\mathList{x}$と小さく$\mathrm{s}$を付けることにする.

\keyword{空リスト}は次のように定義する.\footnote{\haskell では \verb|xs = []| と書く.}
\begin{equation}
\mathList{x}=\mathEmptyList
\end{equation}
任意のリストは次のように構成する.
\begin{equation}
\mathList{x}=x_0:x_1:x_2:\dots:\mathEmptyList
\end{equation}

リストの型はその構成要素の型をブラケットで包んで表現する.\footnote{\haskell では \verb|xs :: [Int]| と書く.}
\begin{equation}
\mathList{x}\mathTypeIs\mathTypeList{\mathTypeInt}
\end{equation}

リストは次のように構成することもできる.\footnote{\haskell では \verb|xs = [1, 2..100]| と書く.}
\begin{equation}
\mathList{x}=\mathMakeList{1,2,\dots,100}
\end{equation}

リストの構成には\keyword{内包表記}が使える.例を挙げる.\footnote{\haskell では次のように書く.
\begin{footcode}
      xs = [x^2 | x <- [1, 2..100], x>50]
\end{footcode}}
\begin{equation}
\mathList{x}=\mathMakeListComplehention{x^2}{x\mathIn\mathMakeList{1,2\dots100},x>50}
\end{equation}

リストとリストをつなぐ場合は\keyword{リスト結合演算子}$\mathListConcat$を用いる.\footnote{\haskell では \verb|zs = xs ++ ys| と書く.}
\begin{equation}
\mathList{z}=\mathList{x}\mathListConcat\mathList{y}
\end{equation}

\section{マップと畳み込み}

リスト$\mathList{x}$の各要素に関数$f$を適用して,その結果をリスト$\mathList{z}$に格納するためには次のように\keyword{マップ演算子}$\mathMap$を用いる.\footnote{\haskell では \verb|zs = f `map` xs| と書く.}
\begin{equation}
\mathList{z}=f\mathMap\mathList{x}
\end{equation}

リスト$\mathList{x}$の各要素を先頭から順番に2項演算子を適用して,その結果を得るには畳み込み演算子を用いる.例えば整数リストの和は次のように書ける.\footnote{\haskell では \verb|z = foldl 0 (+) xs| と書く.}
\begin{equation}
z=\mathFold{0}{(\mathAnonymousParameter+\mathAnonymousParameter)}\mathList{x}
\end{equation}
リスト$\mathList{x}$が$[x_0,x_1,\dots,x_n]$のとき,一般に
\begin{equation}
\mathFold{a}{\mathAnonymousOperator}{\mathList{x}}
=a\mathAnonymousOperator x_0\mathAnonymousOperator x_1\dots x_{n-1}\mathAnonymousOperator x_n
\end{equation}
である.

畳み込み演算子には次の右結合バージョンが存在する.\footnote{\haskell では \verb|foldr| を用いる.}
\begin{equation}
\mathFoldRight{a}{\mathAnonymousOperator}{\mathList{x}}
=a\mathAnonymousOperator\left(x_0\dots\left(x_{n-2}\mathAnonymousOperator\left(x_{n-1}\mathAnonymousOperator x_n\right)\right)\right)
\end{equation}

\section{Maybe}

計算は失敗する可能性がある.例えば
\begin{equation}
z=y/x
\end{equation}
のときに$x\mathCompareEq0$であったとしたら,この計算は失敗する.プログラムが計算を失敗した場合,たいていのプログラマは大域ジャンプを試みる.しかし大域ジャンプは変数の書き換えを行うことであるから,別の方法が望まれる.\haskell では失敗する可能性がある場合にはMaybeという機構が使える.

いま関数$f$が引数$x$と$y$を取り,$x\neq0$であるならば$y/x$を返すとする.もし$x\mathCompareEq0$であれば失敗を意味する$\mathNothing$(ナッシング)を返すとする.すると関数$f$の定義は次のようになる.
\begin{equation}
fyx=\mathIf{x\neq0}{y/x}{\mathNothing}\dots\text{(不完全)}
\end{equation}
残念ながら上式は不完全である.なぜならば$x\neq0$のときの戻り値は数であるのに対して,$x\mathCompareEq0$のときの戻り値は数ではないからである.そこで
\begin{equation}
\mathMonadic{f}yx=\mathIf{x\neq0}{\mathMakeJust{y/x}}{\mathNothing}
\end{equation}
とする.ここに$\mathMakeJust{y/x}$は数$y/x$から作られる,Maybeで包まれた数である.
% ~fの理由.

整数型$\mathTypeInt$をMaybeで包む場合は$\mathTypeMaybe{\mathTypeInt}$と書く.Maybeで包まれた型を持つ変数は$\mathMaybe{x}$のように小さく$?$をつける.例を挙げる.\footnote{\haskell では \verb|xm :: Maybe Int| と書く.}
\begin{equation}
\mathMaybe{x}\mathTypeIs\mathTypeMaybe{\mathTypeInt}
\end{equation}

Maybeで包まれた型を持つ変数は,値を持つか$\mathNothing$(ナッシング)であるかのいずれかである.値をもつ場合は
\begin{equation}
\mathMaybe{x}=\mathMakeJust{1}
\end{equation}
のように書く.\footnote{\haskell では \verb|xm = Just 1| と書く.}

Maybe変数が値を持たない場合は
\begin{equation}
\mathMaybe{x}=\mathNothing
\end{equation}
と書く.\footnote{\haskell では \verb|xm = Nothing| と書く.}

\section{Maybeに対する計算}

Maybe変数に,非Maybe変数を受け取る関数を適用することは出来ない.そこで特別な演算子$\mathFMap$を用いる.\footnote{\haskell では \verb|zm = (+1) <$> xm| と書く.}
\begin{equation}
\mathMaybe{z}=(\mathAnonymousParameter+1)\mathFMap\mathMaybe{x}
\end{equation}
ここに演算子$\mathFMap$は
\begin{align}
\mathMakeJust{fx}&=f\mathFMap\mathMakeJust{x}\\
\mathNothing&=f\mathFMap\mathNothing
\end{align}
と定義される.

\section{型パラメタ}

型をパラメタとして扱うことができる.任意の型を$\mathTypeGeneral{a}$と,ボールド体小文字で書く.ある型$\mathTypeGeneral{a}$の引数を取り,同じ型を返す関数の型は次のように書ける.\footnote{\haskell では \verb|f :: a -> a| と書く.}
\begin{equation}
f\mathTypeIs\mathTypeFunction{\mathTypeGeneral{a}}{\mathTypeGeneral{a}}
\end{equation}

\keyword{型パラメタ}には制約をつけることができる.型の集合を\keyword{型クラス}と呼び,フラクチュール体で書く.たとえば数を表す型クラスは$\mathClassNum$である.型パラメタ$\mathTypeGeneral{a}$が型クラス$\mathClassNum$に属するとき,上述の関数$f$の型注釈は次のようになる.\footnote{\haskell では \verb|f :: Num a => a -> a| と書く.}
\begin{equation}
f\mathTypeIs
  \mathTypeClass{\mathClassNum}
    {\mathTypeGeneral{a}}
    {\mathTypeFunction{\mathTypeGeneral{a}}{\mathTypeGeneral{a}}}
\end{equation}

型クラスは型に制約を与える.

\tobewritten{\texttt{Num a => x :: a} ならば\texttt{x}が持つべき演算子.}

\tobewritten{型クラスの例.}

\section{関手}

型$\mathTypeGeneral{a}$のリストの変数は
\begin{equation}
\mathList{x}\mathTypeIs\mathTypeList{\mathTypeGeneral{a}}
\end{equation}
という型注釈を持つ.これは
\begin{equation}
\mathList{x}\mathTypeIs\mathFunctorTypeGeneral{[]}{\mathTypeGeneral{a}}
\end{equation}
のシンタックスシュガーである.

型$\mathTypeGeneral{a}$型のMaybeの変数は
\begin{equation}
\mathMaybe{x}\mathTypeIs\mathTypeMaybe{\mathTypeGeneral{a}}
\end{equation}
という型注釈を持つ.

普段遣いの関数
\begin{equation}
f\mathTypeIs\mathTypeFunction{\mathTypeGeneral{a}}{\mathTypeGeneral{a}}
\end{equation}
をリスト変数$\mathList{x}$に適用する場合は
\begin{equation}
\mathList{z}=f\mathMap\mathList{x}
\end{equation}
とする.同じく関数$f$をMaybe変数$\mathMaybe{x}$に適用する場合は
\begin{equation}
\mathMaybe{z}=f\mathFMap\mathMaybe{x}
\end{equation}
とする.

リストもMaybeも元の型$\mathTypeGeneral{a}$から派生しており,関数適用のための特別な演算子を持つことになる.そこで,リストやMaybeは\keyword{関手}という型クラスに属する,型パラメタを伴う型であるとする.関手の型クラスを$\mathClassFunctor$で表す.関手型クラスの$\mathTypeGeneral{a}$型の変数を次のように型注釈する.\footnote{\haskell では \verb|xm :: Functor f => f a| と書く.}
\begin{equation}
\mathPure{x}\mathTypeIs
  \mathTypeClass{\mathClassFunctor}
    {\mathClassGeneral{f}}
    {\mathFunctorTypeGeneral{\mathClassGeneral{f}}{\mathTypeGeneral{a}}}
\end{equation}

型クラス$\mathClassFunctor$に属する型は$\mathFMap$演算子を持たねばならない.演算子$\mathFMap$は次の形を持つ.\footnote{In \haskell, \verb|zm = f <$> xm|.}
\begin{equation}
\mathPure{z}=f\mathFMap\mathPure{x}
\end{equation}
演算子$\mathFMap$の型は次のとおりである.
\begin{equation}
\mathAnonymousParameter\mathFMap\mathAnonymousParameter
  \mathTypeIs\mathTypeFunctionII{\left(\mathTypeFunction{\mathTypeGeneral{a}}{\mathTypeGeneral{b}}\right)}
    {\mathFunctorTypeGeneral{\mathClassGeneral{f}}{\mathTypeGeneral{a}}}
    {\mathFunctorTypeGeneral{\mathClassGeneral{f}}{\mathTypeGeneral{b}}}
\end{equation}

もし変数$\mathPure{x}$の型がリストであれば
\begin{equation}
\mathFMap=\mathMap
\end{equation}
であると解釈する.

Function of parametric type with functor class:\footnote{In \haskell, \verb|f :: Functor f => a -> f a|.}
\begin{equation}
f\mathTypeIs
  \mathTypeClass{\mathClassFunctor}
    {\mathClassGeneral{f}}
    {\mathTypeFunction{\mathTypeGeneral{a}}
      {\mathFunctorTypeGeneral{\mathClassGeneral{f}}{\mathTypeGeneral{a}}}}
\end{equation}


Example function application:\footnote{In \haskell, \verb|zm = (+1) <$> Just x|.}
\begin{equation}
\mathPure{z}=(\mathAnonymousParameter+1)\mathFMap\mathMakeJust{x}
\end{equation}

\section{関手としての関数}

\begin{equation}
f\mathTypeIs\mathTypeFunction{\mathTypeGeneral{q}}{\mathTypeGeneral{r}}
\end{equation}

Function as a functor:\footnote{In \haskell, \verb|f :: ((->) r) q|.}
\begin{equation}
f\mathTypeIs\left(\mathTypeFunction{\mathAnonymousTypeParameter}{\mathTypeGeneral{r}}\right)\mathTypeGeneral{q}
=\mathFunctorTypeGeneral{\left(\mathTypeFunction{\mathAnonymousTypeParameter}{\mathTypeGeneral{r}}\right)}
  {\mathTypeGeneral{q}}
\end{equation}

Thus,
\begin{equation}
f_2\mathCompose f_1\equiv f_2\mathFMap f_1
\end{equation}

\begin{align}
\mathId\mathCompose f&=\mathId f=f\\
(h\mathCompose g)\mathCompose f&=((h\mathCompose)\mathCompose(g\mathCompose))f\\
&=h\mathCompose(g\mathCompose f)
\end{align}

\section{アプリカティブ関手}

Pure:\footnote{In \haskell, \verb|zm = pure x|.}
\begin{equation}
\mathPure{z}=\mathMakePure{x}
\end{equation}

Applicative map:\footnote{In \haskell, \verb|zm = f <*> xm|.}
\begin{equation}
\mathPure{z}=\mathPure{f}\mathApplicativeMap\mathPure{x}
\end{equation}
where
\begin{equation}
\mathPure{f}\mathTypeIs
  \mathFunctorTypeGeneral{\mathTypeGeneral{f}}
  {\mathTypeFunction{\mathTypeGeneral{a}}{\mathTypeGeneral{b}}}
\end{equation}

Applicative style:\footnote{In \haskell, \verb|zm = pure (+) <*> xm <*> ym|.}
\begin{equation}
\mathPure{z}
  =\mathMakePure{f}
  \mathApplicativeMap\mathPure{x}
  \mathApplicativeMap\mathPure{y}
\end{equation}
or\footnote{In \haskell, \verb|zm = f <$> xm <*> ym|.}
\begin{equation}
\mathPure{z}
  =f
  \mathFMap\mathPure{x}
  \mathApplicativeMap\mathPure{y}
\end{equation}
or\footnote{In \haskell, \verb|zm = liftA2 f xm ym|. }
\begin{equation}
\mathPure{z}=\mathApplicativeFuncCall{f\,\mathPure{x}\,\mathPure{y}}
\end{equation}


\section{モナド}

Returning \emph{List}.
\begin{equation}
	.
\end{equation}
Returning \emph{Maybe}:\footnote{In \haskell, \verb|f :: Int -> Maybe Int| and \verb|f x = Just x|.}
\begin{align}
f&\mathTypeIs\mathTypeFunction{\mathTypeInt}{\mathTypeMaybe{\mathTypeInt}}\\
fx&=\mathMakeJust{x}
\end{align}
% Applicative.
Returning \emph{monad}:
\begin{align}
f&\mathTypeIs
  \mathTypeFunction{\mathTypeInt}{\mathFunctorTypeGeneral{\mathClassGeneral{m}}{\mathTypeGeneral{a}}}\\
fx&=\mathMakeReturn{x}
\end{align}


Returning monadic value:\footnote{In \haskell, \verb|f :: Monad m => a -> m a|.}
\begin{equation}
f\mathTypeIs
  \mathTypeClass{\mathClassMonad}
    {\mathClassGeneral{m}}
    {\mathTypeFunction{\mathTypeGeneral{a}}{\mathFunctorTypeGeneral{\mathClassGeneral{m}}{\mathTypeGeneral{a}}}}
\end{equation}

Monadic function binding:\footnote{In \haskell, \verb|zm = xm >>= f1 >>= f2|.}
\begin{equation}
\mathPure{z}=\mathPure{x}\mathBindRight f_1\mathBindRight f_2
\end{equation}
where
\begin{align}
f_1&\mathTypeIs\mathTypeFunction{\mathTypeInt}{\mathTypeMaybe{\mathTypeInt}}\\
f_2&\mathTypeIs\mathTypeFunction{\mathTypeInt}{\mathTypeMaybe{\mathTypeInt}}.
\end{align}

Function binding of monadic function and non-monadic function:\footnote{In \haskell,
\begin{footcode}
zm = xm >>= f >>= g'
  where g' w = pure (g w)
\end{footcode}}
\begin{multline}
\mathPure{z}=\mathPure{x}\mathBindRight f\mathBindRight g'\\
\mathWhere{g'w=\mathMakePure{gw}}
\end{multline}
where
\begin{align}
f&\mathTypeIs\mathTypeFunction{\mathTypeInt}{\mathTypeMaybe{\mathTypeInt}}\\
g&\mathTypeIs\mathTypeFunction{\mathTypeInt}{\mathTypeInt}.
\end{align}
Another solution is:
\begin{equation}
\mathPure{z}=(\mathLiftM{g}\mathCompose f)\mathBindRight\mathPure{x}
\end{equation}
where $\mathLiftM{g}$ means \verb|liftM g| in \haskell.\footnote{In \haskell, \verb|zm = (liftM g . f) xm|.}

\section{種}

\begin{equation}
\mathTypeFunction{\mathAnyKind}{\mathAnyKind}
\end{equation}

\section{Data}

Data:\footnote{In \haskell,
\begin{footcode}
data Suit = Spade | Heart | Club | Diamond
\end{footcode}}
\begin{equation}
\mathData{\mathTypeGeneral{Suit}}
  {\mathConstructor{Spade}\mathOr\mathConstructor{Heart}\mathOr\mathConstructor{Club}\mathOr\mathConstructor{Diamond}}
\end{equation}

Data with parameters:\footnote{In \haskell,
\begin{footcode}
data V2 = V2 { x :: Int, y :: Int}
\end{footcode}
or \verb|data V2 = V2 Int Int|.}
\begin{equation}
\mathData{\mathTypeGeneral{V^2}}
  {\mathConstructor{V^2}\left\{x\mathTypeIs\mathTypeInt,y\mathTypeIs\mathTypeInt\right\}}
\end{equation}

\section{型クラスとインスタンス}

\section{IO}

IO example:\footnote{In \haskell, \verb|main = getLine >>= print >> return 0|.}
\begin{equation}
\mathMain=\mathGetLine\mathBindRight\mathPrint\mathNext\mathMakeReturn{0}
\end{equation}

\section{Do構文}

Do notation:\footnote{In \haskell, \verb|z = do {x' <- x; y' <- y; f x'; g y'}|.}
\begin{equation}
\mathPure{z}=\mathDo{x'\leftarrow\mathPure{x};y'\leftarrow\mathPure{y};fx';gy'}
\end{equation}

\dbend

\section{モノイド則}
型$\mathTypeGeneral{a}$の変数$x,y,z\mathTypeIs\mathTypeGeneral{a}$について,特別な変数$\mathIdentity\mathTypeIs\mathTypeGeneral{a}$および二項演算子$\mathAnyBinaryOperator$ただし$x\mathAnyBinaryOperator y\mathTypeIs\mathTypeGeneral{a}$があり,
\begin{align}
\mathIdentity\mathAnyBinaryOperator x&=x\dots\text{(単位元の存在)}\\
(x\mathAnyBinaryOperator y)\mathAnyBinaryOperator z&=x\mathAnyBinaryOperator(y\mathAnyBinaryOperator z)\dots\text{(結合律)}
\end{align}
であるとき,組み合わせ$(\mathTypeGeneral{a},\mathAnyBinaryOperator,\mathIdentity)$をモノイドと呼ぶ.

組み合わせ$(\mathTypeInt,+,0)$や$(\mathTypeInt,\times,1)$はモノイドである.

同じ型から同じ型への1引数関数を改めて$\mathTypeFunction{\mathTypeGeneral{a}}{\mathTypeGeneral{a}}$で表し,特別な変数$\mathIdentity$を関数$\mathId$,二項演算子を$\mathCompose$とすると以下の関係が成り立つ.
\begin{align}
\mathId\mathCompose f&=f\dots\text{(単位元の存在)}\\
(h\mathCompose g)\mathCompose f&=h\mathCompose(g\mathCompose f)\dots\text{(結合律)}
\end{align}
そこで組み合わせ$(\mathTypeFunction{\mathTypeGeneral{a}}{\mathTypeGeneral{a}},\mathCompose,\mathId)$はモノイドであると言える.

\section{関手則}

関手のマップ演算子$\mathFMap$は以下の\keyword{関手則}に従う.
\begin{align}
\mathId\mathFMap\mathPure{x}&=\mathId\mathPure{x}\\
(g\mathCompose f)\mathFMap\mathPure{x}&=((g\mathFMap)\mathCompose(f\mathFMap))\mathPure{x}\\
&=g\mathFMap(f\mathFMap\mathPure{x})
\end{align}

関手則は\keyword{関手(数学)}に由来する.

\keyword{圏}$\mathCategory{C}$の\keyword{対象}を$\mathObject{X}$とする.圏$\mathCategory{D}$の対象は関手(数学)$\mathFunctor{F}$によって対象$\mathObject{X}$と関係づけられる.圏$\mathCategory{C}$における\keyword{射}$f:\mathObject{X}\rightarrow\mathObject{Y}$が$\mathFunctor{F}f:\mathFunctor{F}\mathObject{X}\rightarrow\mathFunctor{F}\mathObject{Y}$に対応し,次の関係を満たす.
\begin{itemize}
\item $\mathObject{X}\in\mathCategory{C}$に対して$\mathFunctor{F}\mathId_{\mathObject{X}}=\mathId_{\mathFunctor{F}\mathObject{X}}$
\item $f:\mathObject{X}\rightarrow\mathObject{Y}$および$g:\mathObject{Y}\rightarrow\mathObject{Z}$に対して$\mathFunctor{F}(g\mathCompose f)=(\mathFunctor{F}g)\mathCompose(\mathFunctor{F}f)$
\end{itemize}

いま
\begin{align}
\mathId_{\mathObject{X}},\mathId_{\mathFunctor{F}\mathObject{X}}&\rightarrow\mathId\\
f\mathFMap&\rightarrow\mathFunctor{F}f
\end{align}
と対応付けると,関手(数学)が満たす法則と関手則は一致する.

\section{アプリカティブ関手則}

アプリカティブ関手のマップ演算子$\mathApplicativeMap$は以下の規則に従う.
\begin{align}
\mathMakePure{\mathId}\mathApplicativeMap\mathPure{x}&=\mathPure{x}\\
\mathMakePure{f}\mathApplicativeMap\mathMakePure{x}&=\mathMakePure{fx}\\
\mathPure{f}\mathApplicativeMap\mathMakePure{x}&=\mathMakePure{\mathAnonymousParameter\mathApply x}\mathApplicativeMap\mathPure{f}\\
\mathMakePure{\mathAnonymousParameter\mathCompose\mathAnonymousParameter}\mathApplicativeMap\mathPure{h}\mathApplicativeMap\mathPure{g}\mathApplicativeMap\mathPure{f}
&=\mathPure{h}\mathApplicativeMap(\mathPure{g}\mathApplicativeMap\mathPure{f})
\end{align}

\section{モナド則}

モナドのマップ演算子$\mathBind$は以下の規則に従う.
\begin{align}
\mathMonadic{f}\mathBind\mathMakeReturn{x}&=\mathMonadic{f}x\\
\mathMakeReturn{\mathAnonymousParameter}\mathBind\mathPure{x}&=\mathPure{x}\\
(\mathMonadic{g}\mathBind\mathMonadic{f})\mathBind\mathPure{x}&=\mathMonadic{g}\mathBind(\mathMonadic{f}\mathBind\mathPure{x})
\end{align}
% つまり,組み合わせ$(\mathTypeFunction{\mathTypeGeneral{a}}{\mathFunctorTypeGeneral{\mathTypeGeneral{m}}{\mathTypeGeneral{a}}},\mathMakePure{\mathAnonymousParameter},\mathBind)$はモノイドである. -- bindの両辺の型が一致しないのでモノイドではない.

次の\keyword{クライスリスター}すなわち
\begin{equation}
\mathKleisliStar{\mathMonadic{f}}=(\mathMonadic{f}\mathBind\mathAnonymousParameter)
\end{equation}
を用いると,モナド則は次のように書き換えられる.
\begin{align}
\left(\mathKleisliStar{\mathMonadic{f}}\right)\mathMakePure{x}&=\mathMonadic{f}x\\
\mathKleisliStar{\left(\mathMakePure{\mathAnonymousParameter}\right)}\mathPure{x}&=\mathPure{x}\\
\mathKleisliStar{\left(\mathKleisliStar{\mathMonadic{g}}\mathMonadic{f}\right)}\mathPure{x}&=\mathKleisliStar{\mathMonadic{g}}\left(\mathKleisliStar{\mathMonadic{f}}\mathPure{x}\right)
\end{align}

% \begin{tikzpicture}[nodes = {text depth = 1ex, text height = 2ex}]
%   \graph{ tex -> dvi -> ps -> pdf,
%   bib -> bbl,
%   bbl -> dvi};
% \end{tikzpicture}


% http://walk.northcol.org/haskell/monads/

% 正格評価
% https://kazu-yamamoto.hatenablog.jp/entry/20110413/1302683869

% returnの削除
% https://qiita.com/mod_poppo/items/8c483951703b5ea0f74e

% STモナド
% https://qiita.com/7shi/items/2e9bff5d88302de1a9e9

\end{document}
